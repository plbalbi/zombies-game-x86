\documentclass{article}

% --------------------------------------------------
% PACKAGES -----------------------------------------
% --------------------------------------------------

% ENCODING
\usepackage[utf8]{inputenc}
\usepackage[spanish]{babel}

% DOCUMENT
\usepackage{geometry} % Page layout
\usepackage{fancyhdr} % Headers

% MATH
\usepackage{amsmath} % Has everything
\usepackage{amssymb} % Math symbols
\usepackage{dsfont} % Domain font
\usepackage{amsthm} % Theorems

% ALGORITHMS
\usepackage{algorithm} % Algorithm floats
\usepackage{algorithmicx} % Macro definitions for algorithm typesetting
\usepackage{algpseudocode} % Pseudocode package built with algorithmicx

% MISC
\usepackage{graphicx} % Images
\usepackage{lastpage} % Reference to last page
\usepackage{enumitem} % List control
\usepackage[draft]{todonotes} % Todo's
\usepackage{hyperref} % Hyperlinks
\usepackage{lipsum} % Lorem Ipsum
\usepackage{caratulaMetNum} % You know it
%\usepackage[firstpage=true]{background} % Background
\usepackage{wallpaper}
\usepackage{eso-pic}

% ---------------------------------------------------
% SET UP --------------------------------------------
% ---------------------------------------------------

% GEOMETRY

\geometry{
	paper=a4paper,
	tmargin=3.5cm,
	bmargin=3.5cm,
	lmargin=3cm,
	rmargin=3cm,
	headheight = 0pt,
	headsep = 40pt,
	footskip = 40pt
}

% SPACING
\setlength{\parskip}{3mm}

% HEADERS
% * 'fancyhead' and 'fancyfoot' writes to all fielsds
% * 'leftmark' and 'rightmark' have the info of two top-level sections
% * 'sectionmark', I don't know
\pagestyle{fancy}
\renewcommand{\sectionmark}[1]{\markboth{#1}{}}
\fancyhead{}
\fancyfoot{}
\rhead{\nouppercase{\leftmark}}
\cfoot{\thepage / \pageref*{LastPage}}

% HYPERLINKS
\hypersetup{
	colorlinks = true,
    urlcolor = blue,
    linkcolor = red,
    citecolor = red
}

% FONTS
 \renewcommand*\rmdefault{iwona} %	Iwona

% ENVIRONMENTS
\newtheorem*{teorema}{Teorema}
\newtheorem*{proposicion}{Proposición}

% ALGORITHMS
\algrenewcommand\algorithmicrequire{\textbf{Requiere}}
\algrenewcommand\algorithmicensure{\textbf{Asegura}}
\floatname{algorithm}{Algoritmo}

%\backgroundsetup{
%contents={\TileSquareWallPaper{10}{zombie_8bit.jpg}},
%position={0,0},
%angle=0,
%scale=1,
%color=black,
%opacity=0.9
%}

\TileSquareWallPaper{10}{zombie_8bit.jpg}

% ---------------------------------------------------
% TIPS ----------------------------------------------
% ---------------------------------------------------

% * pass 'topsep=0pt' to lists to decrease their skips

% ---------------------------------------------------
% MAIN ----------------------------------------------
% ---------------------------------------------------

\begin{document}

	\titulo{Descripción del tp}
	\fecha{\today}
	\materia{Organización del Computador II}
	\titulo{Trabajo Práctico Número 3}
	\subtitulo{Zombi Defence}
	\grupo{Grupo: ``pin-pun-pan-raca-tan-tan''}

	\integrante{Reyes Mesarra, Darío René}{838/15}	{dario6reyes6@yahoo.com.ar}
	\integrante{Pazos Méndez, Nicolás Javier}{709/15}{npazosmendez@gmail.com}
	\integrante{Balbi, Pablo Luis}{707/15}{pablo.l.balbi@gmail.com}

	\palabraClave{Cómo entender nuestro kernel y no morir en el intento - Vol I}

	\maketitle
	
    \ClearWallPaper

	\tableofcontents

	\pagebreak


	\section{Inicialización}
	\vspace{-1cm}
	\begin{flushright}
	\textit{Compuesto por kernel.asm}
	\end{flushright}

	\texttt{kernel.asm} es el punto de entrada al código del kernel. Se encarga de realizar las operaciones de más bajo nivel del mismo, asegurándose de cargar los valores correspondientes en los registros internos para que los mecanismos del procesador (paginación, interrupciones, modo protegido, tareas, etc) funcionen correctamente. También hace una delegación general de trabajo a los demás módulos del kernel para inicializar las estructuras de datos que hagan falta. En resúmen:

	\begin{itemize}[topsep=0pt]
		\item Configurar video
		\item Cargar la GDT y saltar a modo protegido
		\item Llamar a las rutinas que inicializan la MMU
		\item Cargar un mapa de memoria con identity mapping de los primeros 4MB y activar paginación
		\item Llamar a las rutinas que inicializan las TSS de todas las tareas, y cargar un TR válido al contexto actual
		\item Llamar a las rutinas que inicializan la IDT y cargarlas
		\item Configurar el pic para remapear interrupciones externas y activar interrupciones
	\end{itemize}

	Al activarse las interrupciones, se corre por primera vez la rutina de atención del clock, que contiene el código del scheduler. Llegados a este punto, todo se encuentra inicializado. El control pasa al scheduler, las tareas, y las interrupciones generadas por excepciones, o por eventos generados por el usuario.


	\section{Segmentación}
	\vspace{-1cm}
	\begin{flushright}
	\textit{Compuesto por gdt.c, gdt.h}
	\end{flushright}

    En \texttt{gdt.c} se encuentran \textbf{definidos} los descriptores de segmentos principales:
    tanto de código y de datos de nivel 0 (kernel y tarea idle) y 3 (tareas zombis). Como tomamos el
    modelo flat como estrategia de segmentación, todos los segmentos comparten base $0$ y límite
    $623MB$. El problema de la protección del kernel entonces no será resuelto por segmentación,
    sino por paginación. De cualquier forma, es importante que el nivel de privilegio del segmento
    de código sea 3 para las tareas, pues es lo que define el CPL (Current Privilege Level) de las
    tareas, del cual depende el checkeo de privilegios de paginación.

    No tiene interfaz alguna, pues la información adentro de la GDT no se vuelve a tocar, y para
    acceder a la misma sólo hacen falta los selectores correctos, que están definidos como macros en
    \texttt{defines.h}.

    La GDT también contiene las task-gates de las tareas, pero las mismas serán descriptas en la
    sección del mismo nombre.


	\section{Paginación}
	\vspace{-1cm}
	\begin{flushright}
	\textit{Compuesto por kernel.asm, mmu.c y mmu.h}
	\end{flushright}

	\subsection*{Generalidades}

	Este módulo se encarga de todas las rutinas relacionadas con paginación.

	Las páginas físicas que vamos a utilizar van a provenir de un \textit{área libre} en memoria definida en el rango $0x100000 - 0x3fffff$. Vamos a inicializar una variable \textit{prox\_pag\_lib} al comienzo del área libre que apuntará en todo momento a una página libre que podamos utilizar. Al pedir memoria, lo único que haremos será apuntar a la siguiente página, sin preocuparnos por liberar memoria (dado que sólo vamos a necesitar memoria para $40$ tareas zombi como máximo, no va a ser ningún problema).

	Se hará un mapeo de memoria de los primeros 4MB con identity mapping para el kernel y la tarea idle como pide el enunciado.

	Para las tareas zombis, se mapearán los primeros 4MB con identity mapping y DPL supervisor, y las 9 páginas de la posición del mapa en donde se encuentra el zombi a $0x8000000 - 0x8008fff$ con DPL usuario. Esto implica que las tareas normalmente sólo tendrán acceso a las direcciones lineales en $0x8000000 - 0x8008fff$ (con permisos de lectura y escritura sobre posiciones en el mapa). Al producirse la interrupción del scheduler en una tarea, como el segmento de código cambia a uno de nivel 0 (definido en la IDT), las rutinas tienen acceso a las estructuras del kernel.

	\subsection*{Inicialización}

	Hay detalles a explicar en cómo fueron inicializadas los esquemas de memoria de las tareas. Para no desaprovechar memoria, decidimos no pedir memoria para cada zombi nuevo creado. Lo que hicimos fue pedir memoria para las 16 posibles tareas desde un principio. Dicha inicialización la comienza \texttt{tss.c}, que es la que termina guardando las direcciones de memoria de las páginas pedidas.

	La función que se llama para esto es \texttt{mmu\_inicializar\_esquema\_zombi}. Toma como entrada un \textit{jugador} y un \textit{y} que indica dónde está el zombi verticalmente (el x está implicito en el jugador, ya que todos los zombis comienzan siempre en las mismas columas). Lo que hace la función es pedir dos páginas, una para un page directory, y otra para un page table. Para implementar el identity mapping de los primeros 4MB, aprovecha la tabla de páginas del kernel en la dirección $0x28000$ que ya utiliza el esquema de paginación del kernel. La página nueva que pedimos para una page table la usamos para mapear a las direcciones $0x800000- 0x8008fff$.

	Para completar la page table, se delega a la rutina \texttt{mmu\_mapear\_vision\_zombi}, que a su vez aplica \texttt{mmu\_mapear\_pagina} (del enunciado) nueve veces para mapear individualmente las páginas lineales a partir de la $0x8000000$ a las posiciones adyacentes al zombi. La función toma \textit{jugador}, para saber en qué orden mapear las páginas (que cambian según el jugador, de tal forma que el zombi es agnóstico de a qué bando pertenece), un \textit{x}, un \textit{y}, con el cual calcularemos las direcciones físicas del mapa a las que mapearemos y un \textit{cr3} que indicará sobre qué mapa de memoria se aplicará el mapeo. Observar que esta instrucción no inicializa nada, solo mapea las direcciones que componen la `visión' del zombi, con lo cual es una función bastante general. De hecho, la usaremos también para la syscall mover.

	Haremos una mención a la función \texttt{mmu\_inicializar\_zombi} que nunca es usada, si bien está presente por pedido del enunciado. \todo{No entendí... acá se va a mencionar?}


	\section{Tareas}
	\vspace{-1cm}
	\begin{flushright}
	\textit{Compuesto por tss.h y tss.c}
	\end{flushright}

	La mayor responsabilidad del archivo \texttt{tss.c} es la de inicializar las estructuras de las tareas (\texttt{tss\_inicializar}). Cada tarea tiene asociada una variable global que apunta \todo{wut?} contiene su TSS (las tareas zombis, arrays). Para inicializar una tarea se completa su TSS con los valores que corresponden, y se le asigna una task-gate en la GDT.

	Como mencionamos anteriormente, la idea es inicializar todo lo que necesitamos para los zombis en un principio, y reutilizar esos recursos. \texttt{tss\_inicializar\_zombi} toma un jugador y un número de tarea (del 0 al 7), y:

	\begin{itemize}[topsep=0pt]
		\item Pide memoria para el cr3 (con \texttt{mmu\_inicializar\_esquema\_zombi}) y el esp0 (con \texttt{mmu\_prox\_pag\_libre})
		\item Completa la TSS de la tarea (utilizando el cr3 y el esp0)
		\item Agrega la entrada de la GDT (utilizando la dirección de la tss de la tarea)
	\end{itemize}

	Una vez hecho esto para todos los zombis, los \texttt{esp0} y \texttt{cr3} se guardan como variables globales. Esto es para sobrescribir la TSS con los valores iniciales cuando querramos utilizarlo para un zombi nuevo. Esto ocurrirá en la rutina de atención del teclado, en la que al apretar una tecla Shift se lanzará un zombi, y por ende, una nueva tarea. Dicha rutina necesitará acudir a \texttt{tss.c}, llamando a la función \texttt{tss\_refrescar\_zombi}, que toma un jugador y un número de task, y utilizando el cr3 y el esp0 que le corresponden, completa la tss con sus valores iniciales.

	Se dispone de las funciones \texttt{tss\_leer\_cr3} y \texttt{tss\_escribir\_cr3} para modificaciones que querramos hacer en el mapa de memoria de un TSS determinado.


	\section{Interrupciones}
	\vspace{-1cm}
	\begin{flushright}
	\textit{Compuesto por idt.h, idt.c, isr.asm, interrupt.c y interrupt.h}
	\end{flushright}

	Para el archivo \texttt{idt.c} no hubo muchas decisiones de diseño que tomar: se encuentra allí la rutina que inicializa las entradas de la IDT utilizando la macro proporcionada por la cátedra. En \texttt{isr.asm} están definidas todas las rutinas de atención e interrupción: las excepciones arquitecturales, el teclado, el clock, y la syscall mover. Suelen llamar a rutinas de C ubicadas en \texttt{interrupt.c} para programar con más comodidad.

	El clock tiene el código del scheduler, con lo cuál tendrá su propio apartado. Por la manera en la que organizamos el código, la \textit{isr} del syscall mover terminó teniendo muchas responsabilidades, con lo cuál también tendrá una sección aparte.

	\subsection*{Excepciones}

	En general, todas las excepciones se manejan de una forma parecida. Primero nos encargamos de guardar el contexto para una posible llamada a Debug (más información en dicha sección). Después, hay dos variantes: o la excepción fue generada por una tarea zombi, o por el kernel/tarea idle. Para verificar en qué caso estamos, utilizamos el CPL del proceso interrumpido, que se encuentra en el CS guardado en la pila.

    En el primer caso, cuando venimos de una tarea, como pide el enunciado, se `mata' la tarea
    llamando a \texttt{handle\_zombi\_exception} de \texttt{interrupt.c}. Lo que hace esta rutina es
    avisarle al scheduler que el zombi ya no existe, y pintar la pantalla para notificar al usuario.
    El número de `restantes' no baja en esta instancia, sino que baja al lanzar un zombi, asi que no
    hace falta modificar el estado global del juego. Una vez terminado esto, hacemos un task-switch
    a Idle. Nunca vamos a volver a la tarea, por lo que no hace falta completar con más código. Esto
    ocurre porque el scheduler dio de baja el contexto de la tarea, por lo que se lo saca de la
    rotación de tareas. Cuando el scheduler quiere reutilizar el `slot' del zombi, la TSS asociada
    al mismo es reescrita, momento en el cual perdemos el contexto viejo definitivamente. \todo{mmm
    en realidad en la tss por ahí tenes algo más viejo que lo que había cuando se la desalojó, no?}

	El segundo caso, en el que venimos del kernel, marca fallo no planificado. Por dicha razón, decidimos en \texttt{handle\_kernel\_exception} escribir en pantalla datos de la excepción generada (nombre, número de vector, registros varios), y haltear el procesador.

	Ambos casos son muy parecidos, pero hay un detalle que complejizó el código. Ese detalle es que algunas excepciones pushean el Error Code y otras no; esto hace que la lectura de estos datos no pueda hacerse igual para todas ellas. Lo resolvimos pusheando un valor cualquiera para las excepciones que no generan Error Code, de tal manera que a partir de ese entonces podamos utilizar el mismo código para cualquier caso (esto es posible porque no hacemos uso del Error Code).

	\subsection*{Teclado}

	La rutina de atención e interrupción lee inmediatamente el dato proveido por el teclado, y se lo pasa a la rutina \texttt{handle\_keyboard} de \texttt{interrupt.c}. Ésta verifica qué tecla se pulsó, y llama a las rutinas de \texttt{game.c} correspondientes. Por ejemplo, al presionar \texttt{W} se llama a \texttt{game\_mover\_jugador} (pasándole la dirección y el jugador que se movió). También se puede prender y apagar el modo de debug con la tecla $Y$, por lo que esta rutina también lo verifica (y llama a funciones de \texttt{debug.c} para que se encarguen).


	\section{Juego}
	\vspace{-1cm}
	\begin{flushright}
	\textit{Compuesto por game.c y game.h}
	\end{flushright}

	Este módulo contiene las variables que mantienen el estado del juego. Ellas son:

	\begin{itemize}[topsep=0pt]
		\item Las posiciones $y$ de los cursores de los jugadores
		\item Los tipos de zombi actuales que tienen seleccionados los jugadores
		\item Los zombis restantes de los jugadores
		\item Los puntajes de los jugadores
		\item Las posiciones de los zombis activos
		\item Los tipos de los zombis activos
		\item Si el modo debug está activo
		\item Si el debugger ya detectó una excepción
	\end{itemize}

	Las últimas dos se verán en Debug. El resto son variables que tienen lo que necesitamos para la lógica del juego. Siempre que hay un cambio en el juego, como la posición de un zombi, de un jugador, o que se mate o lance un zombi, estas variables deben modificarse para reflejar el estado en el que se encuentra el juego. Por comodidad, están declaradas en \texttt{game.h} y se utilizan directamente en archivos como \texttt{interrupt.c} o \texttt{screen.c}, dependiendo del evento.

	Además, este módulo provee rutinas que representan las diversas acciones dentro del juego. Están íntimamente relacionadas con la \textit{isr} del teclado, que es de donde estas funciones son llamadas dependiendo de qué teclas se presionaron.

	La función más interesante es \texttt{game\_lanzar\_zombi}, que crea una nueva tarea. \todo{Acá estaría bueno mencionar que las tareas activas las tenemos guardadas en un arreglo, así se entiende lo de el lugar y el índice} Primero, se le pregunta al Scheduler si hay un lugar libre (\texttt{sched\_hay\_lugar\_zombi}), y si lo hay, se le pide el mismo (\texttt{sched\_indice\_libre}), además de avisarle que a partir de ahora está activo \texttt{sched\_activar\_zombi}. Se pide el cr3 del TSS asociado a dicho índice (\texttt{tss\_leer\_zombi}), y se pisa el viejo mapeo de las páginas que representan la visión del zombi ($0x8000000 - 0x8008fff$) con un mapeo hacia la posición en la que zombi está siendo inicializado (\texttt{mmu\_mapear\_vision\_zombi}). Además, también se resetea el contexto entero de la tarea con \texttt{tss\_refrescar\_zombi}). Finalmente, se actualizan las variables del juego, y se dibuja en pantalla los cambios realizados.

	\section{Scheduler}
	\vspace{-1cm}
	\begin{flushright}
	\textit{Compuesto por isr.asm, sched.c y sched.h}
	\end{flushright}

    La rutina principal del scheduler en \texttt{isr.asm} depende de
    \texttt{sched\_proximo\_indice}, que le indica al selector la próxima tarea a correr. Si no se
    cargó ninguna tarea todavia ($TR = 0$) o si la próxima tarea es la misma que la actual (porque
    hay una sóla tarea corriendo), entonces no se hace nada (ya que de llevarse a cabo un
    Task-Switch, se produciría caso, fuego y destrucción), retornando de la interrupción. De lo
    contrario, hacemos un task-switch a la nueva tarea mediante un jump far con un selector al
    task-gate indicado (el `próximo índice' que se obtuvo\todo{¿Qué pasaba si hacias un task-switch
    a vos mismo? \textbf{CAOS FUEGO DESTRUCCION}}). El RPL de dicho selector debe ser 0, porque es
    el privilegio que necesitamos para iniciar la conmutación de tareas (el CPL también es 0, porque
    la interrupción del reloj está en un segmento de nivel 0). Un comportamiento especial que posee
    el scheduler, consiste en que si se encuentra prendido el flag de \textit{debug}, entonces 
    \texttt{sched\_proximo\_indice} siempre devoldera el índice de la tarea Idle, \textit{pausando}
    el juego.

	En el archivo \texttt{sched.c}, se encuentran definidas las estructuras que utiliza el scheduler
    para llevar a cabo un seguimiento de que tareas estan corriendo (\textit{tasks\_A} y
    \textit{tasks\_B}), la última tarea ejectuada de cada jugador (\textit{running\_A} y
    \textit{running\_B}), y qué jugador es el próximo al que el scheduler le debe transferir el
    control (\textit{next\_player}). También contiene las runtinas que indican cuál es la siguiente
    tarea a ser ejectuada (mencionada anteriormente), y funciones de interfaz con las estructuras
    del mismo (Para indicar qué tarea esta siendo ejecutada, para agregar nuevas tareas al
    scheduler, o para consultar las mismas).


	% Files Left:
	% isr.asm, sched.c y sched.h -> SCHEDULER
	% interrupt.c -> SYSCALL MOVER
	% screen.c y screen.h -> SCREEN (global)
	% debug.c y debug.h

	\section{Debug}
	\vspace{-1cm}
	\begin{flushright}
		\textit{Compuesto por debug.c y debug.h}
	\end{flushright}
\todo[inline]{Me copé escribiendo y quedó todo en un bloque. Tal vez se puede dividir en subtítulos}
	Estos archivos fuente contienen dos conjuntos de funciones: un primer conjunto de funciones que sirve para imprimir en pantalla errores varios (por ejemplo excepciones del kernel sin resolver), y otro conjunto que compone el funcionamiento del modo debug del juego, que pide la consigna. Las primeras funciones son escencialmente renombres a llamadas de la función \texttt{print} (y sus variantes) con determinados valores de posición y atributos.

	Nótese que el modo debug tiene su utilidad para excepciones de los zombis, y no así las del kernel. Las excepciones del kernel no se manejan de la misma manera (ver Interrupciones), y no dan lugar a este modo.

	Al presionarse la tecla Y se activa el modo debug llamando a la función \texttt{debug\_on}, que lo primero que hace es poner en \textit{true} la variable global del juego \texttt{debug}, indicando que se prendió el modo debug. Este valor de verdad hace que el scheduler no actue, pausando todo movimiento del juego instantáneamente. Luego se procede a guardar los valores de la pantalla que se sobreescribirán al mostrar la ventana; para esto se llama a la función \texttt{backup\_screen}, que copia los datos correspondientes a un arreglo global, al que accederá también la función \texttt{restore\_screen} al salir del modo debug, que lleva a cabo el proceso inverso. Este arreglo está definido en el mismo archivo, por lo que se encuentra en el área de memoria del kernel.

	Luego de haber hecho el \textit{backup} de la pantalla correspondiente, se procede a imprimir la información guardada correspondiente al contexto que provocó la primera excepción. Esta se encuentra almacenada en un \textit{struct} global, cuyos campos son todos los registros de interés. Esta información es guardada por la función \texttt{debug\_save\_context}, que  es llamada por una excepción únicamente si fue la primera en ocurrir; para ello se tiene otro booleano global del juego, \texttt{debug\_hubo\_excepcion}, que comienza en falso y se establece en verdadero cuando ocurre la primera excepción.

	Los valores de la pila del contexto que guarda el modo debug merecen una mención especial. No
    podemos asegurar que el registro \texttt{esp} esté apuntando a lo que la rutina utilizaba como
    pila, pero lo asumiremos así. Sin embargo, tomaremos la precaución de verificar que la dirección
    de memoria que contenga se encuentre en el rango $0x8000000 - 0x8008fff$.\todo{La pila de cada
    tarea no se encuentra mapeada de 0x8001000 para arriba?} Si es así, se muestran tantos valores de 4 \textit{bytes} como indique la distancia entre \texttt{esp} y \texttt{ebp}, con un tope máximo de 10. Si no se encuentra en rango, el modo debug no muestra valores de la pila.

	Si se presiona la tecla Y y el modo debug está prendido, entonces se llama a la función \texttt{debug\_off}, que simplemente restaura la pantalla y establece al booleano \texttt{debug} en false, para que el scheduler retome su funcionamiento normal.


	\section{Syscall}
	\vspace{-1cm}
	\begin{flushright}
		\textit{Compuesto por syscall.h, isr.asm y interrupt.c}
	\end{flushright}
    Los llamados a la syscall se realizan mediante la funcion syscall\_mover definida en syscall.h
    (proveída por la cátedra). Como la syscall es, en otras palabras, una llamada a un servicio del
    kernel, la comunicación entre la tarea llamadora y el kernel se realiza por medio de una
    interrupción (número 102). Una vez realizada la interrupción, la misma es atendida por el
    hanlder de la misma, definido en isr.asm. El mismo realiza una llamada a handle\_syscall\_mover,
    pasandole como parámtro la dirección en la cual avanzar la tarea (zombie).

    En interrupt.c, se enucentra definida la rutina (handle\_syscall\_mover) que lleva a cabo la acción de la syscall (mover
    la tarea en una dirección dada). La misma consiste en:
    \begin{enumerate}
        \item Identificar la tarea que realizo el llamado a la syscall.
        \item Calcular la posición del mapa destino, a la cual se moverá la tarea (zombie).
        \item Replicar el código de la tarea, en la posición destino.
        \item Mapear en el esquema de paginación de la tarea la nueva posicón del código, así como
            sus páginas adyancentes (la cuales son la zona visible por la misma).
        \item Pintar en la pantalla al zombie en su nuvea locación, y dejar un \textit{trace} en su
            anterior.
        \item Si la tarea llego al otro extremo del mapa, actualizar los puntajes, llamar a la
            rutina encargada de destruir la tarea, y en caso de algún jugador haya ganado,
            festejarselo apropiadamente.
    \end{enumerate}
    
    La información que se recupera para identificar la tarea llamadora, se obtiene mediante
    consultas a las estructuras del juego, asi como las del scheduler. Por otro lado, las rutinas
    utilizadas para replicar a la tarea, e interactuar con su esquema de paginación, son las mismas
    que se utilizaron (levemente modificadas en el caso de la replicación, ya que se debe copiar
    también el stack de la tarea) al inicializarlas, vistas en la sección de paginación. En el caso
    de que el zombie haya ganado un punto para su respectivo jugador, los mismos se actualizan en
    variables globales, pertenecientes a la estructura del juego.

\end{document}
